\documentclass[12pt]{article}
\usepackage[utf8]{inputenc}
\usepackage[T1]{fontenc}
\usepackage{lmodern}
\usepackage[english]{babel}
\usepackage{enumitem}
\usepackage{geometry}
\usepackage{graphicx}
\usepackage{booktabs}
\usepackage{url}
\usepackage{hyperref}
\def\UrlBreaks{\do\/\do-\do_}
\usepackage{xurl}
\usepackage{float}
\usepackage{placeins}
\usepackage{caption}
\usepackage{listings}
\usepackage{color}

\geometry{a4paper, margin=1in}

\title{Object-Oriented Programming \\ Semester 2025-III \\ GetClass - Final Delivery}
\author{Alejandro Escobar 20251020094\\ Jhon Gonzalez 20251020087 \\ Sebastián Zambrano 20251020102\\ \\ Computer Engineering Program \\ Universidad Distrital Francisco José de Caldas}
\date{}

\lstset{
  language=Java,
  basicstyle=\ttfamily\small,
  numbers=left,
  numberstyle=\tiny,
  stepnumber=1,
  numbersep=5pt,
  frame=single,
  breaklines=true,
  showstringspaces=false
}

\begin{document}

\maketitle
\thispagestyle{empty}

\newpage
\section{Introduction — Business Model Focus}

GetClasses is a digital marketplace that connects students seeking academic support with qualified tutors. 
The core business hypothesis is that a centralized platform offering verified profiles, scheduling tools, 
and trust mechanisms increases match success, platform revenue, and user retention compared to offline alternatives.

\paragraph{Domain Problem and Target Market}
Students face difficulty finding qualified, available, and affordable tutors. Tutors struggle to reach a wide audience 
and manage schedules efficiently. The target market includes high-school and university students seeking flexible 
academic support and freelance tutors aiming to professionalize their services.

\paragraph{Value Proposition and Business Drivers}
GetClasses provides:
\begin{itemize}
    \item \textbf{Efficient Matching:} filters by subject, availability, rating, and cost.
    \item \textbf{Trust and Credibility:} verified profiles, ratings, and reviews to reduce risk.
    \item \textbf{Operational Simplicity:} tutors manage schedules, sessions, and communication easily.
    \item \textbf{Revenue Generation:} predictable income through booking fees or subscriptions.
\end{itemize}

\paragraph{Business Justification}
The platform reduces search friction, increases booking reliability, and enables a two-sided market, 
directly impacting key metrics: active tutors, conversion rate, and repeat bookings.

\newpage
\section{Technical Design — UML and SOLID}

\subsection*{UML Class Diagram}

\begin{figure}[H]
\centering
\includegraphics[width=0.9\textwidth]{UMLACTU.drawio.png}
\caption{UML class diagram reflecting SOLID principles.}
\end{figure}

\subsubsection*{Key Modeling Decisions}
\begin{itemize}
    \item \textbf{User} — abstract base class with id, name, email, password; authentication behavior encapsulated.
    \item \textbf{TutorUser} — aggregates \texttt{Schedule}, \texttt{Reviews}, and \texttt{TutorInfo}.
    \item \textbf{StudentUser} and \textbf{AdminUser} — specialized subclasses.
    \item \textbf{ClassRequest} — composes \texttt{Request} and \texttt{Answer}, with \texttt{sendRequest()} as primary responsibility.
    \item \textbf{Chat} — associates two \texttt{User} instances and contains a list of \texttt{Message} objects.
\end{itemize}

\subsection*{SOLID Principles — Examples}

\paragraph{Single Responsibility Principle}
\begin{figure}[H]
\centering
\includegraphics[width=0.9\textwidth]{imagen_2025-12-10_191253028.png}
\caption{Code example: Class Review and his constructor.}
\end{figure}

\begin{lstlisting}
// Aggregates reviews; only handles review logic
//Cuando se crea por primera vez
    public Review(UserTeacher teacher, int rate, String comment, UserStudent student){

        this.tutorId = teacher.getId();
        this.studentId = student.getId();
        this.rate = setRate(rate);
        this.comment = setText(comment);
        this.date = LocalDate.now();
    }

    //Cuando se carga de DB
    public Review(int id, int tutorId, int studentId, int score, String comment, LocalDate date){
        this.id = id;
        this.tutorId = tutorId;
        this.studentId = studentId;
        this.rate = score;
        this.comment = comment;
        this.date = date;
    }
\end{lstlisting}


\paragraph{Justification}
Interfaces enable loose coupling (DIP), while classes maintain single responsibility (SRP). 
Liskov Substitution is respected via abstract \texttt{User} and its subclasses. 
Open/Closed is applied to \texttt{ReviewsManager} — new behavior extends class without modifying existing code.

\newpage
\section{Boundaries and Interactions}


\subsubsection*{Narrative}
\begin{itemize}
    \item UI (JavaFX) triggers events; controllers handle domain logic.
    \item Domain services operate on entities and call repositories.
    \item Persistence layer (JSON prototype / SQLite) handles data storage.
    \item External services (Payment, Email) invoked on domain events.
\end{itemize}

\newpage
\section{Requirements — Functional and Non-Functional}

\subsection*{Functional Requirements}
- Student registration, tutor registration, booking, messaging, review creation, schedule management.

\subsection*{Non-Functional Requirements — Quality Attributes}
\begin{enumerate}
    \item \textbf{Performance:} Sub-second response for search operations in local datasets to ensure smooth user experience.
    \item \textbf{Scalability:} Layered architecture allows server-based deployment in the future.
    \item \textbf{Maintainability:} Code is modular, unit-testable, and documented.
    \item \textbf{Security:} Passwords are hashed (BCrypt); sensitive data is encrypted and validated.
    \item \textbf{Reliability:} SQLite transactions ensure atomic writes; JSON fallback allows safe prototyping.
    \item \textbf{Usability:} Standard UI patterns; main tasks achievable in ≤3 clicks.
\end{enumerate}

\newpage
\section{User Stories — Given / When / Then}

\begin{table}[h!]
\centering
\small
\begin{tabular}{|p{3cm}|p{11cm}|}
\hline
\textbf{ID:} 1 & \textbf{Tutor Registration} \\
\hline
\textbf{Priority:} & High \\
\hline
\textbf{Effort (hrs):} & 6 \\
\hline
\textbf{Description:} & As a tutor, I want to register on the platform by entering my personal and professional information so that I can offer tutoring sessions. \\
\hline
\textbf{Acceptance Criteria:} & The tutor account is successfully created after entering valid information. \\
\hline
\end{tabular}
\caption{User Story 1 – Tutor Registration}
\end{table}

\begin{table}[h!]
\centering
\small
\begin{tabular}{|p{3cm}|p{11cm}|}
\hline
\textbf{ID:} 2 & \textbf{Tutor Profile Picture} \\
\hline
\textbf{Priority:} & Medium \\
\hline
\textbf{Effort (hrs):} & 3 \\
\hline
\textbf{Description:} & As a tutor, I want to upload a profile picture so that students can identify me easily. \\
\hline
\textbf{Acceptance Criteria:} & The uploaded image appears correctly on the tutor's profile page. \\
\hline
\end{tabular}
\caption{User Story 2 – Tutor Profile Picture}
\end{table}

\begin{table}[h!]
\centering
\small
\begin{tabular}{|p{3cm}|p{11cm}|}
\hline
\textbf{ID:} 3 & \textbf{Tutor Search} \\
\hline
\textbf{Priority:} & High \\
\hline
\textbf{Effort (hrs):} & 8 \\
\hline
\textbf{Description:} & As a student, I want to search for tutors by subject, rate, or language so that I can find the best match for my learning needs. \\
\hline
\textbf{Acceptance Criteria:} & Tutors matching the selected filters are displayed correctly in the search results. \\
\hline
\end{tabular}
\caption{User Story 3 – Tutor Search}
\end{table}

\begin{table}[h!]
\centering
\small
\begin{tabular}{|p{3cm}|p{11cm}|}
\hline
\textbf{ID:} 4 & \textbf{Tutor Listing} \\
\hline
\textbf{Priority:} & High \\
\hline
\textbf{Effort (hrs):} & 5 \\
\hline
\textbf{Description:} & As a student, I want to view a list of available tutors so that I can compare their profiles and select one. \\
\hline
\textbf{Acceptance Criteria:} & The system displays a complete list of tutors with names, subjects, and ratings. \\
\hline
\end{tabular}
\caption{User Story 4 – Tutor Listing}
\end{table}

\begin{table}[h!]
\centering
\small
\begin{tabular}{|p{3cm}|p{11cm}|}
\hline
\textbf{ID:} 5 & \textbf{Chat with Tutor} \\
\hline
\textbf{Priority:} & High \\
\hline
\textbf{Effort (hrs):} & 6 \\
\hline
\textbf{Description:} & As a student, I want to chat in real time with tutors so that I can clarify doubts before booking a session. \\
\hline
\textbf{Acceptance Criteria:} & The chat allows real-time message exchange between tutor and student. \\
\hline
\end{tabular}
\caption{User Story 5 – Chat with Tutor}
\end{table}

\begin{table}[h!]
\centering
\small
\begin{tabular}{|p{3cm}|p{11cm}|}
\hline
\textbf{ID:} 6 & \textbf{Tutor Rates Student} \\
\hline
\textbf{Priority:} & Medium \\
\hline
\textbf{Effort (hrs):} & 3 \\
\hline
\textbf{Description:} & As a tutor, I want to rate students after a session to maintain platform quality and accountability. \\
\hline
\textbf{Acceptance Criteria:} & The tutor can submit a rating and review after completing a class. \\
\hline
\end{tabular}
\caption{User Story 6 – Tutor Rates Student}
\end{table}

\begin{table}[h!]
\centering
\small
\begin{tabular}{|p{3cm}|p{11cm}|}
\hline
\textbf{ID:} 7 & \textbf{Student Rates Tutor} \\
\hline
\textbf{Priority:} & Medium \\
\hline
\textbf{Effort (hrs):} & 3 \\
\hline
\textbf{Description:} & As a student, I want to rate tutors after a session so that other users can make informed decisions. \\
\hline
\textbf{Acceptance Criteria:} & The student can rate the tutor and leave a comment after class. \\
\hline
\end{tabular}
\caption{User Story 7 – Student Rates Tutor}
\end{table}

\begin{table}[h!]
\centering
\small
\begin{tabular}{|p{3cm}|p{11cm}|}
\hline
\textbf{ID:} 8 & \textbf{View Ratings} \\
\hline
\textbf{Priority:} & Medium \\
\hline
\textbf{Effort (hrs):} & 3 \\
\hline
\textbf{Description:} & As a user, I want to view tutor and student ratings so that I can evaluate trust and performance. \\
\hline
\textbf{Acceptance Criteria:} & Ratings and feedback are visible and linked to corresponding user profiles. \\
\hline
\end{tabular}
\caption{User Story 8 – View Ratings}
\end{table}

\begin{table}[h!]
\centering
\small
\begin{tabular}{|p{3cm}|p{11cm}|}
\hline
\textbf{ID:} 9 & \textbf{Tutor Auto-Responder} \\
\hline
\textbf{Priority:} & Low \\
\hline
\textbf{Effort (hrs):} & 2 \\
\hline
\textbf{Description:} & As a tutor, I want to enable an auto-responder when I am unavailable so that students receive immediate feedback. \\
\hline
\textbf{Acceptance Criteria:} & The system automatically sends a pre-defined message when the tutor is offline. \\
\hline
\end{tabular}
\caption{User Story 9 – Tutor Auto-Responder}
\end{table}

\begin{table}[h!]
\centering
\small
\begin{tabular}{|p{3cm}|p{11cm}|}
\hline
\textbf{ID:} 10 & \textbf{Contact Support/Admin} \\
\hline
\textbf{Priority:} & High \\
\hline
\textbf{Effort (hrs):} & 4 \\
\hline
\textbf{Description:} & As a user, I want to contact platform support or administrators to report issues or request assistance. \\
\hline
\textbf{Acceptance Criteria:} & Support requests are sent successfully and confirmation is received. \\
\hline
\end{tabular}
\caption{User Story 10 – Contact Support/Admin}
\end{table}


\newpage
\section{CRC Cards — Responsibilities (Detailed)}

\begin{table}[H]
\centering
\begin{tabular}{|p{7cm}|p{4cm}|}
\hline
\multicolumn{2}{|c|}{\textbf{Class: ProfileUpdate}} \\ 
\hline
\textbf{Responsibility} & \textbf{Collaborators} \\ 
\hline
Registers and updates the basic profile information for a user; validates input and triggers persistence mechanisms. &
\begin{tabular}[t]{@{}l@{}}
User
\end{tabular} \\ 
\hline
\end{tabular}
\caption{ProfileUpdate CRC — Detailed responsibilities and collaborators}
\end{table}

\begin{table}[H]
\centering
\begin{tabular}{|p{7cm}|p{4cm}|}
\hline
\multicolumn{2}{|c|}{\textbf{Class: User}} \\ 
\hline
\textbf{Responsibility} & \textbf{Collaborators} \\ 
\hline
Represents general user attributes and behaviors; manages authentication, profile information, and basic interactions with system services. &
\begin{tabular}[t]{@{}l@{}}
ProfileUpdate, Teacher, Student
\end{tabular} \\ 
\hline
\end{tabular}
\caption{User CRC — Detailed responsibilities and collaborators}
\end{table}

\begin{table}[H]
\centering
\begin{tabular}{|p{7cm}|p{4cm}|}
\hline
\multicolumn{2}{|c|}{\textbf{Class: TutorInfo}} \\ 
\hline
\textbf{Responsibility} & \textbf{Collaborators} \\ 
\hline
Manages tutor-specific information such as subjects, certifications, hourly rates, and experience; provides data for profile display and business logic. &
\begin{tabular}[t]{@{}l@{}}
Teacher
\end{tabular} \\ 
\hline
\end{tabular}
\caption{TutorInfo CRC — Detailed responsibilities and collaborators}
\end{table}

\begin{table}[H]
\centering
\begin{tabular}{|p{7cm}|p{4cm}|}
\hline
\multicolumn{2}{|c|}{\textbf{Class: StudentInfo}} \\ 
\hline
\textbf{Responsibility} & \textbf{Collaborators} \\ 
\hline
Manages student-specific information such as enrollment, progress, and preferences; provides data for profile and interaction tracking. &
\begin{tabular}[t]{@{}l@{}}
Student
\end{tabular} \\ 
\hline
\end{tabular}
\caption{StudentInfo CRC — Detailed responsibilities and collaborators}
\end{table}

\begin{table}[H]
\centering
\begin{tabular}{|p{7cm}|p{4cm}|}
\hline
\multicolumn{2}{|c|}{\textbf{Class: Teacher}} \\ 
\hline
\textbf{Responsibility} & \textbf{Collaborators} \\ 
\hline
Represents tutor user; inherits attributes and behaviors from User; manages personal schedule, class requests, reviews, and payment interactions. &
\begin{tabular}[t]{@{}l@{}}
User, TutorInfo, Reviews, ClassRequest, Schedule, PaymentSystem
\end{tabular} \\ 
\hline
\end{tabular}
\caption{Teacher CRC — Detailed responsibilities and collaborators}
\end{table}

\begin{table}[H]
\centering
\begin{tabular}{|p{7cm}|p{4cm}|}
\hline
\multicolumn{2}{|c|}{\textbf{Class: Student}} \\ 
\hline
\textbf{Responsibility} & \textbf{Collaborators} \\ 
\hline
Represents student user; inherits attributes and behaviors from User; manages class bookings, schedule interaction, and reviews. &
\begin{tabular}[t]{@{}l@{}}
User, StudentInfo, Reviews, ClassRequest, Schedule, PaymentSystem
\end{tabular} \\ 
\hline
\end{tabular}
\caption{Student CRC — Detailed responsibilities and collaborators}
\end{table}

\begin{table}[H]
\centering
\begin{tabular}{|p{7cm}|p{4cm}|}
\hline
\multicolumn{2}{|c|}{\textbf{Class: PaymentSystem}} \\ 
\hline
\textbf{Responsibility} & \textbf{Collaborators} \\ 
\hline
Handles all payment transactions between students and tutors; validates payment methods, processes refunds, and maintains transaction records. &
\begin{tabular}[t]{@{}l@{}}
Teacher, Student
\end{tabular} \\ 
\hline
\end{tabular}
\caption{PaymentSystem CRC — Detailed responsibilities and collaborators}
\end{table}

\begin{table}[H]
\centering
\begin{tabular}{|p{7cm}|p{4cm}|}
\hline
\multicolumn{2}{|c|}{\textbf{Class: Reviews}} \\ 
\hline
\textbf{Responsibility} & \textbf{Collaborators} \\ 
\hline
Manages creation, modification, and deletion of reviews; aggregates tutor ratings and feedback for system display. &
\begin{tabular}[t]{@{}l@{}}
Review, Student, Teacher
\end{tabular} \\ 
\hline
\end{tabular}
\caption{Reviews CRC — Detailed responsibilities and collaborators}
\end{table}


\newpage
\section{GUI — TutorCard and TutorListPanel}

\subsection*{TutorCard}
\texttt{TutorCard} is a JavaFX \texttt{BorderPane} that visually represents a tutor's profile. Responsibilities:
\begin{itemize}
    \item Display name, age, "about me", subjects (buttons), cost per hour, rating, favorite status.
    \item Emit click events to controller via \texttt{TutorCardListener}.
\end{itemize}

\subsubsection*{Design justification}
\begin{itemize}
    \item Follows SRP: only rendering, no business logic.
    \item Listener interface ensures MVC separation; controller handles interactions.
    \item Modular design supports reuse in \texttt{TutorListPanel} and future extensions.
\end{itemize}

\subsection*{TutorListPanel}
\begin{itemize}
    \item Holds multiple \texttt{TutorCard} instances in a scrollable view.
    \item Responsible for rendering the tutor list and delegating click events.
    \item Maintains separation of concerns: UI container only, logic handled by controller.
\end{itemize}

\begin{figure}[H]
\centering
\includegraphics[width=0.45\textwidth,height=0.3\textheight,keepaspectratio]{MainScreen.png}
\hfill
\includegraphics[width=0.45\textwidth,height=0.3\textheight,keepaspectratio]{ProfileTutorScreen.png}
\caption{GUI mockups: Main (Left) y profile (right).}
\end{figure}

\newpage
\section{Presentation Layer — Architecture and Execution}

\subsection*{Package Organization}
\begin{itemize}
    \item \texttt{com.getclasses.GUI.Controllers} — JavaFX controllers, FXML, views (TutorCard, TutorListPanel).
    \item \texttt{com.getclasses.Classes} — Entities and services (User, ReviewsManager, BookingService).
    \item \texttt{com.getclasses.Database} — Repositories (JSON/SQLite).
    \item \texttt{com.getclasses.GUI} — Repository interfaces, external service adapters.
\end{itemize}

\subsection*{Runtime Flow}
\begin{enumerate}
    \item Application initializes JavaFX and injects repositories into services.
    \item Controllers call services; services manipulate entities and persist via repositories.
    \item UI tasks that may block are executed in \texttt{Task}/\texttt{Service} threads.
\end{enumerate}

\newpage
\section{Persistence — JSON and SQLite Integration}


\subsection*{SQLite Integration}
\begin{lstlisting}
public class ConnectionDB {

    private static final String URL = "jdbc:sqlite:src/main/resources/GetclassDB.db";
    public static Connection getConnection() {
        Connection conn = null;
        try {
            conn = DriverManager.getConnection(URL);
            System.out.println("Conexión establecida con éxito");
        } catch (SQLException e) {
            System.out.println("Error al conectar: " + e.getMessage());
        }
        return conn;
    }
}
\end{lstlisting}

Repositories abstract persistence, enabling UI and domain to remain unchanged when swapping storage.

\newpage
\section{Conclusions}

\begin{itemize}
    \item Architecture respects SOLID and MVC principles.
    \item TutorCard/TutorListPanel designed for reuse and clear separation of concerns.
    \item Persistence mechanisms (JSON/SQLite) fully integrated with domain and UI.
    \item Non-functional requirements are justified with technical reasoning.
    \item Document reflects business model focus and addresses professor's feedback.
\end{itemize}

\newpage
\section{References}
\begin{itemize}
    \item Overleaf. (2024). \textit{LaTeX tutorial: Learn LaTeX step by step}. Retrieved from \url{https://www.overleaf.com/learn}
    \item Lamport, L. (1994). \textit{LaTeX: A Document Preparation System}. Addison-Wesley.
    \item Lucidchart. (2023). \textit{UML Diagram Tutorial}. Retrieved from \url{https://www.lucidchart.com/pages/uml-diagram}
    \item Ambler, S. W. (2023). \textit{Agile Modeling: UML and Class Design}. Retrieved from \url{http://www.agilemodeling.com/artifacts/classDiagram.htm}
    \item Beck, K., \& Fowler, M. (2000). \textit{Planning Extreme Programming}. Addison-Wesley.
    \item Mountain Goat Software. (2022). \textit{User Stories}. Retrieved from \url{https://www.mountaingoatsoftware.com/agile/user-stories}
    \item Coad, P., \& Yourdon, E. (1991). \textit{Object-Oriented Design}. Prentice Hall.
    \item Sommerville, I. (2015). \textit{Software Engineering} (10th ed.). Pearson.
    \item Visual Paradigm. (2023). \textit{CRC Cards Tutorial}. Retrieved from \url{https://www.visual-paradigm.com/guide/uml-unified-modeling-language/what-is-crc-card/}
    \item IEEE. (2023). \textit{Guide to Software Design Documentation (IEEE 1016-2020)}.
\end{itemize}

\end{document}
